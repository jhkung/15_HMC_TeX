\documentclass[pageno]{jpaper}

\usepackage[normalem]{ulem}

\begin{document}

\title{NeuroCube: Low-Power Reconfigurable Neuromorphic Architecture Using Hybrid Memory Cube}
%\author{Emre Ozer & Thomas F. Wenisch}

\date{}
\maketitle

\thispagestyle{empty}

\begin{abstract}
This document serves as a sample for submissions to MICRO 2014.
We provide some guidelines that authors should follow when submitting papers to
the conference.
\end{abstract}

\section{Introduction}

THIS PART IS FOR INTRODUCTION

1. Machine Learning is important application in RMS (Recognition/Mining/Synthesis)

2. However, it's impossible to operate current ML techniques in embedded system due to massive computation

3. As demands for ML in embedded system such as IoT/Mobile platform increases, low-power and high power-efficiency ML is required

4. Therefore, Specific Architecture for ML is required which is better than General Purpose Micro-Arch such as CUDA

5. Deep Learning Network: compsed of multiple different type of NN is powerful tool in ML

6. To operated diff. NNs with single ASIC, Reconfigurbility is important in Neuromorphic Architecture


%This document provides the formatting instructions for submissions to the 47th
%Annual IEEE/ACM International Symposium on Microarchitecture,
%2014~\cite{micro47}. In an effort to respect the efforts of reviewers and in
%the interest of fairness to all prospective authors, we request that all
%submissions to MICRO-47 follow the formatting and submission rules detailed
%below.  Submissions that (grossly) violate these instructions may not be
%reviewed, at the discretion of the program chair, in order to maintain a review
%process that is fair to all potential authors.
%
%An example submission (formatted using the MICRO-47 submission format) that
%contains the submission and formatting guidelines can be downloaded from here:
%\href{http://www.microarch.org/micro47/files/micro47-template.pdf}{Sample
%PDF}. The contents of this document are the same as the contents of the
%submission instructions that appear on
%\href{http://www.microarch.org/micro47/submission.html}{this website}.
%
%All questions regarding paper formatting and submission should be directed to
%the program chair.

\section{Previous Work}
In this section, we will introduce recent machine learning techniques using different types of neural network and hardware implemenation based on ASIC or FPGA. 

\subsection{Neural network for machine learning} 
1. Why do we need different NNs other than Convolutional NNs?

2. How can we classify different NNs

3. What is the characteristic of diff. NNs (pro - cons) or appropriate application

Can we compute number of basic computations such as multiplications or additions for some applications? 

Maybe table to compare diff NNs will be good.
\subsection{Neuromorphic hardware implementation}
text


\section{Reconfigurable Neuromorphic Architecture}
text
\subsection{External memory}
text
\subsection{Cache memory}
text
\subsection{Processing elements}
As we explained before, main basic operations for NNs are multiplication, addition, activation function, sampling ... So we will prepare all functions to cover different neural networks. Then controlling the data from memory to processing elements can implement different neural networks.
\subsection{Network-on-chip}
text

\section{Traffic Analysis on Network-on-Chip}
With conventional DRAM interface (low bandwidth)

\section{Hybrid Memory Cube}
hmc introduce
\subsection{Architecture of HMC}
hmc introduce
\subsection{Intranetwork of HMC through logic die}
hmc introduce
\subsection{Internetwork of HMC through high speed links}
hmc introduce
\subsection{Neurocube using Intranetwork of HMC}
Processor-in-Memory (PIM) design hmc introduce
Area/power/thermal limitation

\section{Traffic Analysis of NeuroCube}
Introduce improvement of NeuroCube with HMC with high bandwidth


\section{Conclusions}
conclusions


\bstctlcite{bstctl:etal, bstctl:nodash, bstctl:simpurl}
\bibliographystyle{IEEEtranS}
\bibliography{references}

\end{document}






%%%%%%%%%%%%%  MICRO TEMPLATE %%%%%%%%%%%%%%%%%
%For MICRO-47, we are instituting a new paper submission format
%and reference requirements. All submissions are allowed a maximum 
%of 11 pages of single-spaced two-column text content according to
%the formatting guidelines below.  In addition, submissions are 
%allowed a references section of unlimited length outside of this
%11-page limit.  

%References must include complete author lists to facilitate
%the reviewing process.	


%If you are using \LaTeX~to typeset your paper, then we suggest
%that you use the template available here:
%\href{http://www.microarch.org/micro47/files/micro47-latex-template.tar.gz}{\LaTeX~Template}.
%(\href{http://www.microarch.org/micro47/files/micro47-template.pdf}{This
%document} was prepared with that template.)  If you are using a different
%software package to typeset your paper, then please adhere to the guidelines
%mentioned in Table~\ref{table:formatting}.
%
%\begin{table}[h!]
%  \centering
%  \begin{tabular}{|l|l|}
%    \hline
%    \textbf{Field} & \textbf{Value}\\
%    \hline
%    \hline
%    Page limit & 11 pages + references\\
%    \hline
%    Paper size & US Letter 8.5in $\times$ 11in\\
%    \hline
%    Top margin & 1in\\
%    \hline
%    Bottom margin & 1in\\
%    \hline
%    Left margin & 0.75in\\
%    \hline
%    Right margin & 0.75in\\
%    \hline
%    Space between columns & 0.25in\\
%    \hline
%    Body font & 10pt\\
%    \hline
%    Abstract font & 10pt, italicized\\
%    \hline
%    Section heading font & 12pt, bold\\
%    \hline
%    Subsection heading font & 10pt, bold\\
%    \hline
%    Caption font & 9pt, bold\\
%    \hline
%    References & 8pt, no page limit, list all authors\\
%    \hline
%  \end{tabular}
%  \caption{Formatting guidelines for submission.}
%  \label{table:formatting}
%\end{table}

%\textbf{Please ensure that you include page numbers with your
%submission}. This makes it easier for the reviewers to refer to
%different parts of your paper when they provide comments.


%
%\noindent\textbf{\sout{Author List.}} All submissions are double
%blind. Therefore, please do not include any author names in the
%submission. You must also ensure that the metadata included in the
%PDF does not give away the authors. If you are improving upon your
%prior work, refer to your prior work as a third person and include
%references to your past work. 
%
%\noindent\textbf{Figures and Tables.} Ensure that the figures and
%tables are legible.  Please also ensure that you refer to your
%figures in the main text. Many reviewers print the papers in
%gray-scale. Therefore, if you use colors for your figures, ensure
%that the different colors are distinguishable in gray-scale.
%
%\noindent\textbf{Main Body.} Avoid bad page or column breaks in
%your main text, i.e., last line of a paragraph at the top of a
%column or first line of a paragraph at the end of a column. If you
%begin a new section or sub-section near the end of a column,
%ensure that you have at least 2 lines of body text on the same
%column. Note that the entire main body of your submission as well as any
%Figures, footnotes, etc., must conform to the 11-page content limit;
%only references may appear on additional pages.
%
%\noindent\textbf{References.} There is no length limit for references. 
%Each reference must explicitly list all authors of the paper. 
%Papers not meeting this requirement will be rejected. Authors of NSF 
%proposals should be familiar with this requirement. Knowing all
%authors of related work will help find the best reviewers.
%
%\section{Submission Instructions}
%
%\subsection{Paper Authors}
%
%Declare all the authors of the paper upfront. Addition/removal of authors once
%the paper is accepted will have to be approved by the program chair.
%
%\subsection{Conflict Responsibilities}
%
%Authors must register all their conflicts on the paper submission site. 
%Conflicts are needed to ensure appropriate assignment of reviewers. If a paper
%is found to have an undeclared conflict that causes a problem OR if a paper 
%is found to declare false conflicts in order to abuse or "game" the review 
%system, the paper may be rejected. 
% 
%Please declare a conflict of interest (COI) with the following for any author of your paper:
%
%\begin{enumerate}
%\item Your Ph.D. advisor, post-doctoral advisor, and Ph.D. students
%\item Other past or current advisors
%\item Current or past students
%\item Family relations by blood or marriage (if they might be potential reviewers)
%\item People with the same affiliation
%\item People whom you co-authored accepted/rejected/pending papers with in the last 5 years
%\item People whom you co-authored accepted/pending grant proposals with in the last 5 years
%\end{enumerate}
%
%"Service" collaborations like co-authoring a CSTB report or co-presenting tutorials do 
%not constitute conflicts. However, there may be others not covered 
%by the above with whom you know a COI exists. Please report such COIs; however, you 
%will need to justify them. Please be reasonable. For example, just because a reviewer
%works on similar topics as your paper, you cannot declare a COI with that reviewer. We
%will carefully check the justification of conflicts and take action when authors seem to be 
%blacklisting reviewers without sufficient justification. 
%
%We hope to draw most reviewers 
%from the PC and the ERC, but others from the community may also write reviews. Please
%declare all your conflicts (not just restricted to the PC and ERC). When in doubt, 
%contact the program chair.
%
%
%\subsection{Concurrent Submissions and Resubmissions of Already Published Papers}
%
%By submitting a manuscript to MICRO-47, the authors guarantee that the
%manuscript has not been previously published or accepted for
%publication in a substantially similar form in any conference or
%journal. The authors also guarantee that no paper which contains
%significant overlap with the contributions of the submitted paper is
%under review to any other conference or journal or workshop, or will
%be submitted to one of them during the MICRO-47 review
%period. Violation of any of these conditions will lead to rejection.
%
%Extended versions of papers accepted to IEEE Computer Architecture
%Letters can be submitted to MICRO-47.
%As always, if you are in doubt, it is best to contact the program chair. 
%
%\section{Submission Site}
%
%A link to the submission site will be posted on the MICRO-47 web site
%\href{http://www.microarch.org/micro47/}{http://www.microarch.org/micro47/}
%in mid April. 



